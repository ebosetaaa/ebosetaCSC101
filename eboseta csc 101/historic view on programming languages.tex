\documentclass{article}
 
 \title{HISTORIC VIEW ON 5 PROGRAMMING LANGUAGES}
 \date{20-10-2021}
 \author{Ejola Eboseta}
 
 \begin{document}
 	\pagenumbering{gobble}
 	\maketitle
 	\newpage
 	\pagenumbering{arabic}
 	\section{$\underline{WHAT ARE PROGRAMMING LANGAUGES?}$}
 \paragraph{A programming language is a package or set of instructions that a computer uses to perform specific tasks
 }
\paragraph{There are varieties of programming language which are:}
\subparagraph{Python,
	HTML,
	JAVA,
	PHP,
	Cobol,
	C++,
	CSS,
	Basic etc
	}
\newpage
\section{PYTHON}
\subsection{Python is a programming language that was made by Guido van Rossum in 1980. The project kicked off in 1989. The name of the programming language was inspired by a skit called "Monty Python's Flying Circus." He was a great admirer of the comedy and wanted a name that was both simple and distinctive.
	Although python is a high-level programming language it is way easier to use than the likes of c++ or Java and this is because of the lesser procedures that it requires to code in python.
	A python code was released in 1991 with modula-3, the version was called 0.9.0
	.The first python was created to enable programmers to use a more friendly programming language that allows you to code with fewer codes or instructions and with the aim of code readability. Python is now adopted by scientists, accountants, and non-programmers.
	Many operating system developers support the use of python it is also used by famous organizations
	The creator of this programming language also worked on several programming languages like ABC programming language etc.
	There are lots of apps that python can design some not even known to us. For instance, entertainment applications like YouTube, Netflix, Spotify, etc. It also designs console applications and gaming applications e.g. Sims 4, Civilization, etc. 
	Python uses several IDE’s(Integrated Development Environment) such as Kite, Pycharm, Spyder.
	Python is similar to other programming languages like Lisp, Haskell, Smalltalk Perl, C++, and Java.
}
\newpage
\section{HTML}
\subsection{Sir Tim Berners-Lee invented the HTML (HyperText Markup Language) computer language in 1991.
	Throughout that time, it has evolved from a simple language with a limited number of tags to a complex mark-up system that allows authors to create all-singing, all-dancing Web pages with animated graphics, music, and a variety of gimmicks. HTML consists of HTML+, HTML 2, HTML 3.2, and HTML 4.
	HTML 1 was the initial version of the markup language. This early HTML was quite different from the HTML we use today, and it had a lot more limitations. Only a few individuals engaged in web development at the time, and the internet was not extensively used. Those interested in web development, on the other hand, couldn't do much more than post some basic text on the web with HTML (the original HTML consisted of only 22 tags)
	Tim Berners-Lee, a physicist, and professor was the primary author of HTML, he created the World Wide Web in 1989 at CERN with the aid of his colleagues. He was designated one of Time magazine's 100 most important people of the twentieth century for this feat.
	HTML is used to create internet navigation, web document creation, web pages development, responsive images on web pages, etc. 
	HTML uses an IDE(Integrated Development Environment) called Visual Studio Code.
}
\newpage
\section{JAVA}
\subsection{Java was an object-oriented programming language created in the early 1990’s by James Gosling. The team launched this effort to establish a language for digital devices like set-top boxes and televisions. C++ was originally offered for use in the project, but it was rejected for a variety of reasons (e.g., C++ required more RAM).
	The name Java was chosen after considerable deliberation due to its distinctiveness. The name Java is derived from a kind of espresso bean known as Java. Gosling came up with the term while having coffee near his office.
	James Gosling a.k.a. ‘Dr Java’ was born May 19 1955. He is a computer scientist and the founder of the programming language Java.
	Java, including the compiler and virtual machine, was built from the ground up by him. For this feat, he was admitted into the National Academy of Engineering of the United States. He's also made important contributions to NeWS and Gosling Emacs, among other software projects.
	Java is used to create mobile applications, desktop GUI applications, gaming applications, business applications etc. 
	Java uses several IDE’s(Integrated Development Environment) like Kite, Eclipse, NetBeans etc. 
}
\newpage
\section{PHP}
\subsection{PHP(Hypertext PreProcessor) was created in 1994 by Rasmus Lerdorf.
	Rasmus, who is now a Yahoo! developer, was looking for a way to make it easier to create content for his website, something that would work with HTML but also provide him more power and flexibility. He required an easy method to create scripts that would run on his web server and generate content, as well as handle data returned to the server from the web browser.
	PHP is used to create web-based applications, ecommerce  applications etc.
	PHP uses  IDE’s (Integrated Development Environment) like NetBeans, Eclipse, PHPStorm etc.  
}
\newpage
\section{COBOL}
\subsection{COBOL(COmmon Business-Oriented Language)was created in 1959 by Grace Hopper and her team. This programming language is used in applications of main frame computers
	In the spring of 1959, a two-day meeting called the Conference on Data Systems Languages brought together computer specialists from business and government (CODASYL). Many of Hopper's former workers sat on the short-term group that established the new language, and she acted as a technical consultant to the committee COBOL(COmmon Business-Oriented Language)
	COBOL creates business-oriented applications.
	COBOL uses an IDE(Integrated Development Environment) called OpenCobolIDE
}
\newpage
\
 \end{document}